% Created 2020-11-06 Jum 20:24
% Intended LaTeX compiler: pdflatex
\documentclass[11pt]{article}
\usepackage[utf8]{inputenc}
\usepackage[T1]{fontenc}
\usepackage{graphicx}
\usepackage{grffile}
\usepackage{longtable}
\usepackage{wrapfig}
\usepackage{rotating}
\usepackage[normalem]{ulem}
\usepackage{amsmath}
\usepackage{textcomp}
\usepackage{amssymb}
\usepackage{capt-of}
\usepackage{hyperref}
\author{TC}
\date{2020-11-01}
\title{TS-setting}
\hypersetup{
 pdfauthor={TC},
 pdftitle={TS-setting},
 pdfkeywords={},
 pdfsubject={},
 pdfcreator={Emacs 26.3 (Org mode 9.1.9)}, 
 pdflang={English}}
\begin{document}

\maketitle
\tableofcontents


\section{Cara mengetahui service running}
\label{sec:org0721654}
Untuk bisa mengetahui service yang sedang berjalan ada bebarapa perintah
bisa dilakukan 
\subsection{Mengetahui running untuk TS-Receiver}
\label{sec:orgd4d92ee}
Pada dasarnya setiap socket server yang berjalan untuk menerima data dari
perangkat merupakan sebuah script yang menjalankan sebuah aplikasi biner 
yang terdapat dalam folder Trumon. Nama file biner tersebut adalah TcpServerTS1.
Untuk bisa mengetahui running TS-Receiver anda bisa menggunakan perintah ps\footnote{ps atau process status merupakan program untuk mengetahui program yang sedang berjalan
link referensi \url{https://en.wikipedia.org/wiki/Ps\_(Unix)}}.
\begin{verbatim}
ps -axu | grep TcpServerTS1
\end{verbatim}
Untuk mengecek service yang barjalan pada bash anda bisa menggunakan perintah 
berikut. 
\begin{verbatim}
ps -aux | grep start_Recv4421.sh
ps -aux | grep start_Parser.sh
ps -aux | grep PM.sh
\end{verbatim}
\end{document}
